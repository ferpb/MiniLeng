
\vspace*{1cm}
\section*{\hfil Resumen \hfil}
\addcontentsline{toc}{section}{Resumen}

En este informe de documenta el desarrollo de un compilador para el lenguaje MiniLeng durante la asignatura Procesadores de Lenguajes.

MiniLeng es un lenguaje procedimental, estructurado y fuertemente tipado, cuya sintaxis está inspirada en Pascal y C.

El lenguaje es \it{case-insensitive} y soporta comentarios tanto de una sola línea como multilínea. Se pueden declarar variables tanto globales como locales de tres tipos de dato: entero, carácter y booleano, y permite la declaración de acciones (procedimientos) anidadas con paso de parámetros tanto por valor como por referencia.

Se pueden utilizar las estructuras de control ``mientras que'' (while) y ``selección'' (if), y se soporta la asignación de valores a variables y la evaluación de expresiones aritméticas y lógicas.

El lenguaje tiene dos procedimientos predefinidos: ``escribir'', que permite escribir por pantalla variables simples y cadenas de caracteres constantes y ``leer'', que permite leer valores introducidos por el usuario y asignarlos a variables. También tiene dos funciones que permiten transformar un entero a carácter y viceversa: ``entacar'' y ``caraent''.

Por último, se ha añadido la posibilidad de trabajar con vectores unidimensionales. Se permite la declaración de vectores, la asignación de valores a las componentes, la asignación directa entre vectores del mismo tamaño, el uso de componentes en expresiones y el paso de componentes y vectores completos como argumentos al invocar una acción.

Un ejemplo de programa de MiniLeng válido semánticamente se puede ver en el Código~\ref{ejemplo}.

%%%%%% INSERTAR PROGRAMA
\begin{minileng}[caption={Ejemplo de programa en MiniLeng.},label={ejemplo}]
programa factorialVect;
    entero i, r, v[8];

    accion factorial(val entero n; ref entero r);
    principio
        si n < 2 ent
            r := 1;
        si_no
            factorial(n - 1, r);
            r := n * r;
        fsi
    fin

principio
    i := 0;
    mq i < 8
        factorial(i, v[i]);
        escribir("v[", i, "] = ", v[i], "\n");
        i := i + 1;
    fmq
fin\end{minileng}


Cada vez que se produce un error de compilación, el compilador muestra un mensaje explicativo junto al token más relevante en el error y su número de fila y columna en el programa. Distingue entre errores léxicos, sintácticos y semánticos. Cuando el compilador encuentra un error léxico termina la ejecución, pero cuando el error es sintáctico o semántico, intenta seguir compilando hasta el final del fichero para mostrar la máxima información posible sobre los errores que puede en el programa. También se incluyen avisos de compilación, como \textit{underflow}/\textit{overflow} y divisiones por cero.

Si no se han producido errores durante la compilación del programa (no se tienen en cuenta los avisos), se genera un fichero con el mismo nombre del fichero de entrada y extensión \tt{.code} con el código intermedio que contiene las instrucciones para ejecutar el programa en la máquina P. La generación de código se realiza utilizando el esquema AST, que da flexibilidad a la generación de código y permite introducir optimizaciones interesantes.

Para implementar el compilador, se ha utilizado JavaCC, un metacompilador de código abierto para el lenguaje Java.
Se ha desarrollado principalmente, sobre Eclipse 2020-06 con el plugin JavaCC Eclipse Plug-in 1.5.33.
\pagebreak
