\section{Gramática BNF de MiniLeng} \label{gramatica}
En este anexo se incluye la gramática BNF que se ha utilizado en la implementación del compilador. El código de JavaCC que la implementa se encuentra en el fichero \url{src/analizador/minilengcompiler.jj}. Encima de cada regla no terminal del fichero \tt{.jj} se ha añadido un comentario con la regla BNF correspondiente para facilitar la lectura.

\subsection{Tokens}

\begin{Verbatim}[fontsize=\small]
<DEFAULT> SKIP : {
" "
| "\r"
| "\t"
| "\n"
| <"%" ~["%"]> : COMENTARIO
| "%%" : MULTICOMENTARIO
}

<COMENTARIO> SKIP : {
"\n" : DEFAULT
}

<COMENTARIO> MORE : {
<~[]>
}

<MULTICOMENTARIO> SKIP : {
"%%" : DEFAULT
}

<MULTICOMENTARIO> MORE : {
<~[]>
}

<DEFAULT> TOKEN : {
  <tPROGRAMA: "programa"> : {}
| <tVAR: "var"> : {}
| <tPRINCIPIO: "principio"> : {}
| <tFIN: "fin"> : {}
| <tSI: "si"> : {}
| <tENT: "ent"> : {}
| <tSI_NO: "si_no"> : {}
| <tFSI: "fsi"> : {}
| <tMQ: "mq"> : {}
| <tFMQ: "fmq"> : {}
| <tESCRIBIR: "escribir"> : {}
| <tLEER: "leer"> : {}
| <tENTACAR: "entacar"> : {}
| <tCARAENT: "caraent"> : {}
| <tACCION: "accion"> : {}
| <tVAL: "val"> : {}
| <tREF: "ref"> : {}
}

<DEFAULT> TOKEN : {
  <tENTERO: "entero"> : {}
| <tBOOLEANO: "booleano"> : {}
| <tCARACTER: "caracter"> : {}
}

<DEFAULT> TOKEN : {
  <tLLAVE_IZQ: "{"> : {}
| <tLLAVE_DER: "}"> : {}
| <tPARENTESIS_IZQ: "("> : {}
| <tPARENTESIS_DER: ")"> : {}
}

<DEFAULT> TOKEN : {
  <tCORCHETE_IZQ: "["> : {}
| <tCORCHETE_DER: "]"> : {}
}

<DEFAULT> TOKEN : {
  <tOPAS: ":="> : {}
| <tFIN_SENTENCIA: ";"> : {}
| <tSEP_VARIABLE: ","> : {}
| <tMAS: "+"> : {}
| <tMENOS: "-"> : {}
| <tPRODUCTO: "*"> : {}
| <tDIVISION: "/"> : {}
| <tMOD: "mod"> : {}
| <tDIV: "div"> : {}
| <tAND: "and"> : {}
| <tOR: "or"> : {}
| <tNOT: "not"> : {}
| <tMAYOR: ">"> : {}
| <tMENOR: "<"> : {}
| <tIGUAL: "="> : {}
| <tMAI: ">="> : {}
| <tMEI: "<="> : {}
| <tNI: "<>"> : {}
}

<DEFAULT> TOKEN : {
  <#DIGITO: ["0"-"9"]>
| <#LETRA: ["a"-"z"]>
| <tTRUE: "true"> : {}
| <tFALSE: "false"> : {}
| <tIDENTIFICADOR: (<LETRA> | "_" (<LETRA> | <DIGITO>)) (("_")? (<LETRA> | <DIGITO>))*> : {}
| <tCONSTENTERA: (["0"-"9"])+> : {}
| <tCONSTCHAR: "\"" (~["\""])? "\""> : {}
| <tCONSTCAD: "\"" (~["\""])* "\""> : {}
}
\end{Verbatim}

\subsection{No terminales}

\begin{Verbatim}[fontsize=\small]
programa ::= <tPROGRAMA> identificador fin_sentencia declaracion_variables
             declaracion_acciones bloque_sentencias <EOF>

fin_sentencia ::= <tFIN_SENTENCIA>
sep_variable ::= <tSEP_VARIABLE>
parentesis_izq ::= <tPARENTESIS_IZQ>
parentesis_der ::= <tPARENTESIS_DER>
corchete_izq ::= <tCORCHETE_IZQ>
corchete_der ::= <tCORCHETE_DER>
longitud_const ::= <tCONSTENTERA>

principio ::= <tPRINCIPIO>
fin ::= <tFIN>
identificador ::= <tIDENTIFICADOR>
declaracion_variables ::= ( declaracion fin_sentencia )*
declaracion ::= tipo_variables identificadores
tipo_variables ::= ( <tENTERO> | <tCARACTER> | <tBOOLEANO> )
identificadores ::= identificador_declaracion ( sep_variable identificador_declaracion )*
identificador_declaracion ::= identificador ( corchete_izq longitud_const corchete_der )?

declaracion_acciones ::= ( declaracion_accion )*
declaracion_accion ::= cabecera_accion fin_sentencia declaracion_variables
                       declaracion_acciones bloque_sentencias
cabecera_accion ::= <tACCION> identificador ( parametros_formales )?
parametros_formales ::= parentesis_izq ( parametros ( fin_sentencia parametros )* )? parentesis_der
parametros ::= clase_parametros tipo_variables lista_parametros
lista_parametros ::= identificador_parametro ( sep_variable identificador_parametro )*
clase_parametros ::= ( <tVAL> | <tREF> )
identificador_parametro ::= identificador ( corchete_izq longitud_const corchete_der )?

bloque_sentencias ::= principio lista_sentencias fin
lista_sentencias ::= sentencia ( sentencia )*
sentencia ::= ( leer | escribir | identificacion | seleccion | mientras_que )
leer ::= <tLEER> parentesis_izq lista_asignables parentesis_der fin_sentencia
lista_asignables ::= lista_expresiones
escribir ::= <tESCRIBIR> parentesis_izq lista_escribibles parentesis_der fin_sentencia
lista_escribibles ::= escribible ( sep_variable escribible )*
escribible ::= ( expresion | <tCONSTCAD> )
mientras_que ::= <tMQ> expresion lista_sentencias <tFMQ>
seleccion ::= <tSI> expresion <tENT> lista_sentencias ( <tSI_NO> lista_sentencias )? <tFSI>
identificacion ::= identificador ( ( corchete_izq expresion corchete_der )?
                   asignacion fin_sentencia | ( argumentos )? fin_sentencia )
asignacion ::= <tOPAS> expresion
argumentos ::= parentesis_izq ( lista_expresiones )? parentesis_der

lista_expresiones ::= expresion ( sep_variable expresion )*
expresion ::= expresion_simple ( operador_relacional expresion_simple )*
operador_relacional ::= ( <tIGUAL> | <tMENOR> | <tMAYOR> | <tMAI> | <tMEI> | <tNI> )
expresion_simple ::= termino ( operador_aditivo termino )*
operador_aditivo ::= ( <tMAS> | <tMENOS> | <tOR> )
termino ::= factor ( operador_multiplicativo factor )*
operador_multiplicativo ::= ( <tPRODUCTO> | <tDIVISION> | <tMOD> | <tDIV> | <tAND> )
factor ::= ( <tMENOS> factor | <tMAS> factor | <tNOT> factor
             | parentesis_izq expresion parentesis_der
             | <tENTACAR> parentesis_izq expresion parentesis_der
             | <tCARAENT> parentesis_izq expresion parentesis_der
             | identificador ( corchete_izq expresion corchete_der )?
             | <tCONSTENTERA> | <tCONSTCHAR> | <tTRUE> | <tFALSE> )
\end{Verbatim}
