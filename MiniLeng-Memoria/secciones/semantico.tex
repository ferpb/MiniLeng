\documentclass[../main.tex]{subfiles}

\begin{document}
\section{Análisis semántico}
\subsection{Gestión de errores y avisos (\it{warnings}) semánticos}
% Detallarlos errores y avisos (warnings) que se han implementado

\subsection{Controles sobre los tipos de datos}
% Detallar y explicar los controles que se han implementado sobre los tipos de datos (desbordamientos, booleanos, etc.)

\subsection{Propagación y manejo de valores constantes en las expresiones}
% Detallar y explicar cómo se ha implementado la propagación y el manejo de valores constantes en las expresiones.

\subsection{Comprobaciones sobre parámetros \tt{VAL} y \tt{REF}}
% Detallar y explicar cómo se han implementado las comprobaciones sobre parámetros VAL/REF. ¿Dónde se realizan estas comprobaciones? Adjuntar ejemplos que validen la funcionalidad implementada.


No se van a usar parámetros ocultos. Como estamos trabajando con objetos Java, las acciones pueden
tener una lista de referencias a los parámetros. Los parámetros se borrarán de la tabla en el
momento que la función deje de ser visible y se borre.

\end{document}
