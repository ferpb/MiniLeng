\documentclass[../main.tex]{subfiles}

\begin{document}
\section{Mejoras introducidas}
% Detallar las mejoras y puntos opcionales que se han desarrollado sobre la propuesta inicial.
% Especificar detalles de implementación, decisiones de diseño y pruebas/ejemplos de las funcionalidades añadidas.
% Separar las mejoras en bloques, según correspondan al análisis léxico, semántico o a la generación de código.

\subsection{Mejoras en el comportamiento del compilador}

Errores en el uso del programa

\begin{itemize}
\item Si se utiliza una opción inválida o no se especifican correctamente, se muestra el siguiente error:
\begin{codigo}{text}
MiniLeng: Opción inválida <error>
\end{codigo}

\item Si el fichero a compilar no tiene extensión \tt{.ml}, se muestra el error:
\begin{codigo}{text}
MiniLeng: El fichero a compilar tiene que tener extensión .ml
          Fichero introducido: '...'
\end{codigo}

\item Si no se encuentra el fichero a compilar, se indica de la siguiente forma:
\begin{codigo}{text}
MiniLeng: No se ha encontrado el fichero '...'
\end{codigo}

\end{itemize}

Al final de la compilación se muestra un recuento de los errores encontrados:

\begin{codigo}{text}
Errores léxicos: ...
Errores sintácticos: ...
Veces activado panic mode: ...
\end{codigo}

Y se muestra

\begin{codigo}{text}
No se ha podido compilar el programa
\end{codigo}

si ha habido errores durante la compilación, o:

\begin{codigo}{text}
Compilado sin errores
\end{codigo}

si la compilación ha sido exitosa.

Si durante la compilación se ha activado el modo pánico, se muestra el siguiente mensaje:

\begin{codigo}{text}
Se ha activado el panic mode durante la compilación. Corrige los errores y vuelve a compilar.
\end{codigo}

para indicar al usuario que tiene que hace falta volver a compilar el programa.



\subsection{Mejoras en el análisis léxico}
La opción tokens muestra por pantalla los nombres de los tokens que se van reconociendo conforme se analiza el programa. Ejemplo:

\begin{codigo}{text}
tPROGRAMA
tIDENTIFICADOR (Valor: mi_programa)
tFIN_SENTENCIA
tENTERO
tIDENTIFICADOR (Valor: n)
tFIN_SENTENCIA
...
tFIN
\end{codigo}

\subsection{Mejoras en el análisis semántico}
  - Se permiten acciones anidadas con cualquier nivel de profuncidad. Las declaraciones de acciones
    anidadas deben entre la zona de declaración de las variables y la palabra 'principio' de la
    función padre:

  \begin{verbatim}
      accion accion1;
        % Declaración de variables de accion1
        accion accion2;
          % Declaración de variables de accion2
          accion accion3;
            % Declaración de variables de accion3
          principio
            % Sentencias de accion3
          fin
        principio
          % Sentencias de accion2
        fin
      principio
        % Sentencias de accion1
      fin
  \end{verbatim}

  - Se permiten bloques 'seleccion' y 'mientras que' anidados con cualquier nivel de profundidad:

  \begin{verbatim}
      SI <condicion> ENT
        % Sentencias
        SI <condicion> ENT
          % Sentencias
        SI_NO
          % Sentencias
        FSI
        % Sentencias
      SI_NO
        % Sentencias
      FSI

      MQ <condicion>
        % Sentencias
        MQ <condicion>
          % Sentencias
        FMQ
        % Sentencias
      FMQ
  \end{verbatim}

  - Cualquier bloque ('principio/fin', 'seleccion' o 'mientras que') debe contener al menos una
    sentencia.




\subsubsection{Panic Mode}
El compilador dispone de la opción panic (\tt{-p}), que activa el modo pánico durante la compilación. Se entra en este modo cada vez que se produce un error sintáctico porque se esperaba un punto y coma. El analizador descarta todos los tokens siguientes al error hasta que encuentra un carácter \car{;} o el fichero se acaba, entonces sale del modo y sigue analizando la entrada.

Cuando se entra en el modo pánico, se informa al usuario con las siguientes líneas:

\begin{codigo}{text}
MiniLeng: ERROR SINTÁCTICO (línea ..., columna ...): Token incorrecto: '...'. Se esperaba ';'
PANIC MODE: Iniciado panic mode
\end{codigo}

Y se va mostrando en una línea diferente cada token descartado:

\begin{codigo}{text}
> PANIC MODE: Token descartado: '...'
\end{codigo}

Cuando se sale del modo, se muestra:

\begin{codigo}{text}
  > PANIC MODE (línea ..., columna ...): Se ha encontrado ';'
PANIC MODE: Terminado panic mode
\end{codigo}

\subsection{Mejoras en la generación de código}

\end{document}
