\documentclass[../main.tex]{subfiles}

\begin{document}
\section{Análisis sintáctico}

\subsection{Modificaciones sobre la gramática propuesta y decisiones de diseño}
% Modificaciones sobre la gramática propuesta y decisiones de diseño adoptadas.
En la gramática propuesta, algunas reglas estaban incompletas, a continuación se detalla como se han completado y las decisiones de diseño adoptadas.

\begin{description}
\item[Parametros formales] Contiene la definición de los parámetros de una acción, esta estructura es opcional por lo que se ha utilizado el símbolo \car{?}:
\begin{verbatim}
parametros_formales	::=	( parentesis_izq ( lista_parametros )? parentesis_der )?

lista_parametros	::=	parametros ( fin_sentencia parametros )*

\end{verbatim}


\item[Lista de sentencias]

lista_sentencias	::=	sentencia ( sentencia )*

sentencia	::=	( leer | escribir | identificacion | seleccion | mientras_que )


leer	::=	<tLEER> parentesis_izq lista_asignables parentesis_der fin_sentencia

lista_asignables	::=	identificadores

escribir	::=	<tESCRIBIR> parentesis_izq lista_escribibles parentesis_der fin_sentencia
lista_escribibles	::=	lista_expresiones




seleccion	::=	<tSI> expresion <tENT> lista_sentencias ( <tSI_NO> lista_sentencias )* <tFSI>



\item[Expresiones]

lista_expresiones	::=	expresion ( sep_variable expresion )*

expresion	::=	expresion_simple ( operador_relacional expresion_simple )?
operador_relacional	::=	( <tIGUAL> | <tMENOR> | <tMAYOR> | <tMAI> | <tMEI> | <tNI> )
expresion_simple	::=	( <tMAS> | <tMENOS> )? termino ( operador_aditivo termino )*
operador_aditivo	::=	( <tMAS> | <tMENOS> | <tOR> )
termino	::=	factor ( operador_multiplicativo factor )*
operador_multiplicativo	::=	( <tPRODUCTO> | <tDIVISION> | <tMOD> | <tAND> )

\end{description}

Los símbolos \car{,}, \car{;}, \car{(} y \car{)}  se han sustituido en la gramática por sus tokens correspondientes: \tok{tSEP\_VARIABLE}, \tok{tFIN\_SENTENCIA}, \tok{tPRENTESIS\_IZQ} y \tok{tPARENTESIS\_DER}, de esta forma sería posible cambiar los caracteres en un futuro de forma sencilla.

\subsection{Gestión de errores y avisos (\it{warnings}) sintácticos}
% Detallar y explicar el manejo de excepciones y la función de error sintáctico implementada.

\end{document}