\section{Pruebas realizadas}
En esta sección se detalla la metodología seguida para verificar el funcionamiento del compilador desarrollado en cada fase. Para cada fase se incluyen algunos tests junto a sus resultados, no se tratan de unas pruebas intensivas pero sirven para dar una idea del proceso que se ha seguido en la fase de pruebas.

\subsection{Analizador léxico}
El objetivo del analizador léxico es que reconozca todos los tokens de la entrada y que se lancen errores léxicos cuando se encuentran caracteres no soportados. Para probarlo, se intentó compilar varios programas diferentes activando la opción \it{tokens} (\tt{-t}, \tt{-{}-tokens}), que muestra los tokens que se van reconociendo en en análisis, para verificar que todos son reconocidos correctamente.

\subsubsection{Test 1}
% Poner un programa un poco complicado y ver que todos los tokens salen bien reconocidos
Vamos a considerar que tenemos el siguiente programa:

\begin{codigo}[style=minileng,numbers=left]
programa factorial;
entero n, solucion;

accion fact (val entero numero; ref entero solucion);
entero solRecur;
principio
	si numero <= 1 ent
		solucion := 1;
	si_no
	  	fact(numero - 1, solRecur);
		solucion := solucion * solRecur;
	fsi
fin

principio
	n := 10;
	fact(n, solucion);
	escribir(n, "! es: ", solucion);
fin
\end{codigo}

Al compilar el programa utilizando la opción \tt{-t}, se obtiene la siguiente salida, que si se observa detenidamente contiene todos los tokens del programa:

\noindent\begin{minipage}{.45\textwidth}
\begin{codigo}
tPROGRAMA
tIDENTIFICADOR (Valor: factorial)
tFIN_SENTENCIA
tENTERO
tIDENTIFICADOR (Valor: n)
tSEP_VARIABLE
tIDENTIFICADOR (Valor: solucion)
tFIN_SENTENCIA
tACCION
tIDENTIFICADOR (Valor: fact)
tPARENTESIS_IZQ
tVAL
tENTERO
tIDENTIFICADOR (Valor: numero)
tFIN_SENTENCIA
tREF
tENTERO
tIDENTIFICADOR (Valor: solucion)
tPARENTESIS_DER
tFIN_SENTENCIA
tENTERO
tIDENTIFICADOR (Valor: solRecur)
tFIN_SENTENCIA
tPRINCIPIO
tSI
tIDENTIFICADOR (Valor: numero)
tMEI
tCONSTENTERA (Valor: 1)
tENT
tIDENTIFICADOR (Valor: solucion)
tOPAS
tCONSTENTERA (Valor: 1)
tFIN_SENTENCIA
tSI_NO
tIDENTIFICADOR (Valor: fact)
tPARENTESIS_IZQ
tIDENTIFICADOR (Valor: numero)
\end{codigo}
\end{minipage}\hfill
\begin{minipage}{.45\textwidth}
\begin{codigo}
tMENOS
tCONSTENTERA (Valor: 1)
tSEP_VARIABLE
tIDENTIFICADOR (Valor: solRecur)
tPARENTESIS_DER
tFIN_SENTENCIA
tIDENTIFICADOR (Valor: solucion)
tOPAS
tIDENTIFICADOR (Valor: solucion)
tPRODUCTO
tIDENTIFICADOR (Valor: solRecur)
tFIN_SENTENCIA
tFSI
tFIN
tPRINCIPIO
tIDENTIFICADOR (Valor: n)
tOPAS
tCONSTENTERA (Valor: 10)
tFIN_SENTENCIA
tIDENTIFICADOR (Valor: fact)
tPARENTESIS_IZQ
tIDENTIFICADOR (Valor: n)
tSEP_VARIABLE
tIDENTIFICADOR (Valor: solucion)
tPARENTESIS_DER
tFIN_SENTENCIA
tESCRIBIR
tPARENTESIS_IZQ
tIDENTIFICADOR (Valor: n)
tSEP_VARIABLE
tCONSTCAD (Valor: "! es: ")
tSEP_VARIABLE
tIDENTIFICADOR (Valor: solucion)
tPARENTESIS_DER
tFIN_SENTENCIA
tFIN
\end{codigo}
\end{minipage}

\begin{codigo}
\end{codigo}

\subsubsection{Test 2}
% Comprobar que se producen errores
Cuando se encuentra algún carácter no soportado por el lenguaje se produce un error léxico, a no ser que este se encuentre en un comentario, sea una constante carácter o forme parte de una cadena constante.

Si se renombra a \tt{solución} la variable \tt{solución} modificando la linea 2 por la siguiente:

\begin{codigo}[style=minileng]
entero n, solución
\end{codigo}

aparece el error léxico:

\begin{codigo}
ERROR LÉXICO (2, 16): Caracter no reconocido: `ó' (\u00f3)
\end{codigo}

\subsection{Analizador sintáctico}
Para probar el analizador sintáctico una buena estrategia es intentar compilar programas léxicamente correctos pero con errores sintácticos y comprobar que se el compilador lanza errores en los lugares adecuados.

\subsubsection{Test 1}
Tenemos el siguiente sencillo programa, al que le falta un punto y coma en la línea 4.

\begin{codigo}[style=minileng,numbers=left]
programa prueba_sintactico;
entero n;
principio
    n := 1 + 1
    escribir("Hola");
fin
\end{codigo}

Cuando se intenta compilar, el analizador léxico pasa al analizador sintáctico el token \tt{escribir} cuando se necesitaba \car{;}, por lo que aparece el siguiente error:

\begin{codigo}
ERROR SINTÁCTICO (7, 1): Token incorrecto: `escribir'. Se esperaba `;'

Errores sintácticos: 1

No se ha podido compilar el programa.
\end{codigo}

\subsubsection{Test 2}
En la siguiente declaración de acción no se han especificado la clase del primer parámetro:

\begin{codigo}[style=minileng,numbers=left,firstnumber=4]
accion miAccion(entero n, val caracter c);
principio
    escribir(n, c);
fin
\end{codigo}

Al compilar, se ve como el analizador sintáctico intenta seguir compilando el programa después de detectar el primer error, aunque esto lleve a más errores sintácticos:

\begin{codigo}
ERROR SINTÁCTICO (4, 17): Token incorrecto: `entero'. Se esperaba `('
ERROR SINTÁCTICO (4, 17): Token incorrecto: `entero'. Se esperaba `;'
ERROR SINTÁCTICO (4, 27): Token incorrecto: `val'. Se esperaba un identificador
ERROR SINTÁCTICO (4, 27): Token incorrecto: `val'. Se esperaba `;'
ERROR SINTÁCTICO (4, 27): Token incorrecto: `val'. Se esperaba el delimitador de principio de bloque: `principio'
ERROR SINTÁCTICO (4, 27): Token incorrecto: `val'. Se esperaba una sentencia
ERROR SINTÁCTICO (4, 27): Token incorrecto: `val'. Se esperaba el delimitador fin de bloque: `fin'
ERROR SINTÁCTICO (4, 27): Token incorrecto: `val'. Se esperaba el delimitador de principio de bloque: `principio'
ERROR SINTÁCTICO (4, 27): Token incorrecto: `val'. Se esperaba una sentencia
ERROR SINTÁCTICO (4, 27): Token incorrecto: `val'. Se esperaba el delimitador fin de bloque: `fin'
ERROR SINTÁCTICO (4, 27): Token incorrecto: `val'. La declaración del programa es incorrecta

Errores sintácticos: 11

No se ha podido compilar el programa.
\end{codigo}

\subsection{Analizador semántico}
Para probar el analizador semántico se siguió una estrategia similar a la del sintáctico. Se intentan compilar programas sintácticamente correctos pero con errores semánticos y se verifica que los errores aparezcan en los lugares esperados.

\subsubsection{Test 1}
El siguiente programa intenta declarar dos variables con el mismo nombre:

\begin{codigo}[style=minileng,numbers=left]
programa prueba_semantico;
entero n, n;

    accion miAccion
    entero n;
    principio
        escribir(n);
    fin

principio
    n := 1;
    escribir(n);
fin
\end{codigo}

Al intentar compilarlo se obtiene lo siguiente:

\begin{codigo}
ERROR SEMÁNTICO (2, 11): Símbolo: `n'. No se puede redefinir el símbolo
AVISO (5, 8): Símbolo: `n'. Este símbolo, definido en el nivel 1, va a ocultar a otro definido con el mismo nombre en el nivel 0

Errores semánticos: 1
Avisos: 1

No se ha podido compilar el programa.
\end{codigo}

El compilador detecta que se ha intentado declarar una variable dos veces en la línea 2. Dentro de la acción también se declara una variable llamada \tt{n}, pero al estar en un nivel inferior no se está redefiniendo la variable anterior, si no ocultándola.

\subsubsection{Test 2}
En el siguiente programa se intenta operar con tipos diferentes:

\begin{codigo}[style=minileng,numbers=left]
programa prueba_semantico;
entero n;
booleano b;
principio
    b := true or false;
    n := "a" + b;
fin
\end{codigo}

por lo que el compilador genera:

\begin{codigo}
ERROR SEMÁNTICO (6, 14): Símbolo: `+'. El operando 1 debe ser entero

Errores semánticos: 1

No se ha podido compilar el programa.
\end{codigo}

El término que contiene la suma devuelve un \tt{RegistroExpr} de tipo \tt{ENTERO}, por lo que el error no se propaga hasta la asignación.


\subsection{Generación de código}
Para probar la generación de código, se compilaron varios programas semánticamente correctos y se analizó el código generado. Este código también se probó a ejecutar en Hendrix utilizando el ensamblador e intérprete proporcionados. En esta fase ha sido de gran utilidad generar comentarios con la función de cada grupo de instrucciones, para saber que parte del código intermedio corresponde con cada parte del programa MiniLeng.

\subsubsection{Test 1}
El siguiente programa contiene una sentencia mientras que:

\begin{codigo}[style=minileng,numbers=left]
programa generacion_codigo;
entero i;
principio
    i := 10;
    mq i < 10
        escribir(i);
        i := i - 1;
    fmq
fin
\end{codigo}

Analizando el código generado se ve que es correcto y leyendo los comentarios se puede ver la correspondencia con cada parte del programa inicial.

\begin{codigo}
; Programa: generacion_codigo
	ENP L2
; Comienzo de programa: generacion_codigo
L2:
; Asignación a i
; Obtener dirección de VARIABLE ENTERO:          i [0, 3]
	SRF 0 3
; Sustituida expresion constante
	STC 10
	ASG
; Mientras que
	JMP L0
L1:
; Escribir entero
; Obtener valor de VARIABLE ENTERO:          i [0, 3]
	SRF 0 3
	DRF
	WRT 1
; Asignación a i
; Obtener dirección de VARIABLE ENTERO:          i [0, 3]
	SRF 0 3
; Obtener valor de VARIABLE ENTERO:          i [0, 3]
	SRF 0 3
	DRF
; Sustituida expresion constante
	STC 1
	SBT
	ASG
L0:
; Condicion mientras que
; Obtener valor de VARIABLE ENTERO:          i [0, 3]
	SRF 0 3
	DRF
; Sustituida expresion constante
	STC 10
	LT
	JMT L1
; Fin de generacion_codigo
	LVP
\end{codigo}

\subsubsection{Test 2}
El siguiente programa contiene varias expresiones constantes:

\begin{codigo}[style=minileng,numbers=left]
programa generacion_codigo;
entero e;
booleano b;
principio
    e := (20 + 4) * 5 / 4;
    b := (2 * 100) = (100 * 2);
fin
\end{codigo}

El compilador es capaz de calcularlas correctamente y almacenar sus valores directamente sobre la pila:

\begin{codigo}
; Programa: generacion_codigo
	ENP L0
; Comienzo de programa: generacion_codigo
L0:
; Asignación a e
; Obtener dirección de VARIABLE ENTERO:          e [0, 3]
	SRF 0 3
; Sustituida expresion constante
	STC 30
	ASG
; Asignación a b
; Obtener dirección de VARIABLE BOOLEANO:        b [0, 4]
	SRF 0 4
; Sustituida expresion constante
	STC 1
	ASG
; Fin de generacion_codigo
	LVP
\end{codigo}

\subsubsection{Test 3}
% Poner el código generado al acceder a una componente de vector a través de una variable en un parámetro por referencia
A continuación, se va a intentar acceder a una componente de vector que se ha pasado por referencia a una acción para comprobar que el código generado en este caso es correcto.

\begin{codigo}[style=minileng,numbers=left]
programa generacion_codigo;
entero v[10];

accion miAccion(ref entero v[10]);
principio
    escribir(v[1]);
fin

principio
    v[1] := 100;
    miAccion(v);
fin
\end{codigo}

Se ha generado el siguiente código. Se puede comprobar que es correcto y que sigue los aspectos de la generación de código explicados en las secciones anteriores.

\begin{codigo}
; Programa: generacion_codigo
	ENP L2
; Acción: miAccion
L0:
; Recuperar parametro 1: PARAMETRO REF ENTERO:     v[10] [1, 3]
	SRF 0 3
	ASGI
; Comienzo de acción: miAccion
; Escribir entero
; Obtener valor de PARAMETRO REF ENTERO:     v[10] [1, 3]
	SRF 0 3
	DRF
; Indice
; Sustituida expresion constante
	STC 1
	PLUS
	DRF
	WRT 1
	CSF
; Comienzo de programa: generacion_codigo
L2:
; Asignación a v
; Obtener dirección de VARIABLE ENTERO:          v[10] [0, 3]
; Indice: 1
	SRF 0 4
; Sustituida expresion constante
	STC 100
	ASG
; Apilar argumento 1: PARAMETRO REF ENTERO:     v[10] [1, 3]
; Obtener dirección de VARIABLE ENTERO:          v[10] [0, 3]
	SRF 0 3
; Invocar a ACCION:                   miAccion(v) [0, 0]
	OSF 14 0 L0
; Fin de generacion_codigo
	LVP
\end{codigo}
