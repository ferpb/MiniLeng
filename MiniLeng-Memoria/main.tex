\documentclass[a4paper,10pt]{article}

\usepackage{url}

\usepackage[utf8]{inputenc}
\usepackage[spanish, es-tabla]{babel}
\usepackage[indent]{parskip}

\clubpenalty=10000
\widowpenalty=10000

\usepackage[gen]{eurosym}

\usepackage{xcolor}
\usepackage{setspace}
\usepackage{float}
\usepackage{biblatex}
\addbibresource{fuentes.bib}
\setlength{\parindent}{0cm}
\usepackage{amsmath, amssymb}

% Listados de código
\usepackage{minted}
\usepackage{caption}
\newenvironment{code}{\captionsetup{type=listing}}{}
% \SetupFloatingEnvironment{listing}{name=Source Code}
\renewcommand{\listingscaption}{Código}
\BeforeBeginEnvironment{minted}{\medskip}
\AfterEndEnvironment{minted}{\medskip}

\usepackage{csquotes}

\renewcommand{\it}[1]{\textit{#1}}
\renewcommand{\bf}[1]{\textbf{#1}}
\renewcommand{\tt}[1]{\texttt{#1}}

\newcommand{\car}[1]{`\texttt{#1}'}
\newcommand{\tok}[1]{\texttt{\textless#1\textgreater}}

%%%%%%% Comandos para escribir los comandos %%%%%%%%%%%
%% \com{el comando}    para comandos en la misma línea

%% begin{comm}           para comentarios en la misma línea
%%   comandos
%% end{comm}

%% \tt  es una abreviación para no tener que poner \texttt al usar fuente monoespaciada
%%%%%%%%
\newcommand{\com}[1]{\fbox{\mintinline{bash}{$ #1}}}

\newcommand{\BashFancyFormatLine}{%
  \def\FancyVerbFormatLine##1{\$ ##1}%
}
\newcommand{\BashFancyFormatLineDB}{%
  \def\FancyVerbFormatLine##1{SQL> ##1}%
}
\newenvironment{comm}
 {
    \VerbatimEnvironment
    \begin{minted}[frame=single, formatcom=\BashFancyFormatLine, fontsize=\normalsize, xleftmargin=10pt, xrightmargin=10pt]{bash}}
 {\end{minted}}
\newenvironment{commdb}
 {
    \VerbatimEnvironment
    \begin{minted}[frame=single, formatcom=\BashFancyFormatLineDB, fontsize=\normalsize, xleftmargin=21pt, xrightmargin=21pt]{bash}}
 {\end{minted}}
 
 
 
 \newenvironment{codigo}[1]
 {
    \captionsetup{type=listing}
    \VerbatimEnvironment
    \begin{minted}[fontsize=\normalsize, xleftmargin=5pt, xrightmargin=5pt, frame=single, framesep=5pt]{#1}}
    %% Se puede utilizar framesep=10pt para añadir separación entre el marco
    %% Se puede utilizar frame=single para poner una caja alrededor del texto.
    %% frame=lines
 {\end{minted}}


\usepackage{booktabs} % Allows the use of \toprule, \midrule and \bottomrule in tables for horizontal lines
\usepackage{graphicx}
\graphicspath{{imagenes/}{../imagenes/}} % Declara dos paths, uno con respecto a este documento y otro respecto a la carpeta secciones

\usepackage[margin=3cm, headheight=13.6pt]{geometry}
 \usepackage{fancyhdr}
\fancyhf{}
\lhead{\emph{Procesadores de Lenguajes -- 2019-20}}
\rhead{\emph{Compilador de MiniLeng}}
\fancyfoot[C]{\thepage}

\usepackage[titletoc, title]{appendix}

\usepackage{subfiles} % Permite incluir ficheros .tex en el texto

\begin{document}

\begin{titlepage}
    \centering
    \includegraphics[width=1 cm]{logoUZ.jpg}
    
    \textsc{\large Universidad de Zaragoza}
    \rule{\textwidth}{1.6pt}\vspace*{-\baselineskip}\vspace*{2pt} % Thick horizontal rule
    \rule{\textwidth}{0.4pt} % Thin horizontal rule
    
    \vfill
    
    {\LARGE \scshape Procesadores de Lenguajes}
                
    \vspace{2cm}            

    {\bfseries \Huge Compilador de MiniLeng}
    
    \vspace{.5cm} 
    
    {\bfseries \Large Informe sobre el desarrollo del compilador}
    
    \vspace{3cm}    
    
   

    % {\large \scshape Autor}

    %\vspace{0.2cm}
    
    {\large Fernando Peña Bes (NIA: 756012)}
    
    \vfill
    
    \large{Zaragoza, España}
    
    {Curso 2019\,--\,2020}

    \vfill

    \includegraphics[width=5.0cm]{EINA.png}
   
\end{titlepage}

\pagenumbering{Roman}


\vspace*{2cm}
\section*{\hfil Resumen \hfil}
En este informe de documenta el desarrollo de un compilador para el lenguaje MiniLeng durante la asignatura Procesadores de Lenguajes.

MiniLeng es un lenguaje procedural, estructurado y fuertemente tipado. Su sintaxis está inspirada en Pascal.

Para implementar el compilador, se ha utilizado JavaCC, un metacompilador de código abierto para el lenguaje Java. Se ha desarrollado, principalmente, sobre Eclipse 2020-06 con el plugin JavaCC Eclipse Plug-in 1.5.33

\addcontentsline{toc}{section}{Resumen}


\pagebreak

\begin{spacing}{0.1}
\tableofcontents
\end{spacing}
%\listoffigures
%\listoftables

\pagebreak

\setcounter{page}{1}
\pagenumbering{arabic}

\pagestyle{fancy}

\subfile{secciones/lexico}
\subfile{secciones/sintactico}
\subfile{secciones/tabla-simbolos}
\subfile{secciones/semantico}
\subfile{secciones/generacion-codigo}
\subfile{secciones/mejoras-introducidas}
\subfile{secciones/pruebas}

\renewcommand\appendixname{Anexo}
\begin{appendices}
\subfile{secciones/gramatica}
\subfile{secciones/uso}
\end{appendices}

\clearpage{}
\newpage
\nocite{*}
\addcontentsline{toc}{section}{Referencias}
\printbibliography



\end{document}
